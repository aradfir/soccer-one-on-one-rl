\chapter{مقدمه}
\section{مقدمه}
ربوکاپ\LTRfootnote{RoboCup}
 مجموعه‌ای از مسابقات است که به هدف ترویج و پیشرفت علم رباتیک و هوش مصنوعی به صورت سالانه برگزار می‌شود.
در این مسابقات تیم‌های مختلفی از سراسر دنیا شرکت می‌کنند و با استفاده از ربات‌های خود در مسابقات مختلفی همچون ربات‌های انسان‌نما، ربات‌های چرخ‌دار، و لیگ‌های شبیه‌سازی به رقابت می‌پردازند.
از طریق این مسابقات، تیم‌ها می‌توانند تکنیک‌های جدیدی را برای حل مسائل مختلف در علم رباتیک و هوش مصنوعی ارائه کنند و از تجربیات دیگر تیم‌ها بهره‌مند شوند.

یکی از لیگ‌های معروف ربوکاپ،‌ لیگ شبیه‌ساز دو بعدی فوتبال\LTRfootnote{2D Soccer Simulation} است
که محیطی چندعامله با دید ناقص\LTRfootnote{Partial Observation}
 را برای تیم‌ها فراهم می‌کند.
تمرکز این لیگ بیشتر در راستای تصمیم‌گیری در محیط‌های چندعامله و هماهنگی بین عوامل است.
در این لیگ بازیکنان با کمک حسگر‌های مجازی، سایر بازیکنان و توپ را مشاهده می‌کنند و بر اساس این اطلاعات باید تصمیم‌گیری‌های مناسبی انجام دهند.
تصمیم‌های عامل می‌توانند از نوع حرکت، ضربه زدن به توپ، تکل زدن، چرخاندن بدن و گردن باشد.
در فصل‌های آتی به توضیحات بیشتری در مورد این لیگ و نحوه عملکرد آن خواهیم پرداخت.

این محیط با توجه به چالش‌هایی که دارد و از سوی دیگر به علت تکرار پذیری و ابزار‌هایی که در اختیار می‌گذارد، یک محیط مناسب برای تست و ارزیابی الگوریتم‌های هوش مصنوعی و یادگیری تقویتی است.
از این رو انجام پژوهش‌های مختلف در این محیط می‌تواند به پیشرفت علمی و فناوری کمک کند.
\section{تعریف مساله}
در این پروژه قصد داریم از یکی از محیط‌های موجود در لیگ شبیه‌سازی دو بعدی فوتبال استفاده کنیم و با چند الگوریتم متفاوت 
مسئله تک به تک با دروازه‌بان را حل کنیم.
سپس راه حل‌های ارائه شده را با یکدیگر مقایسه کرده و به نتیجه‌گیری‌هایی در مورد عملکرد هر الگوریتم و تاثیر پارامتر‌های مختلف آنها بر عملکرد می‌رسیم.

در این پروژه قصد داریم عامل مهاجم را به صورت یک الگوریتم یادگیری تقویتی آموزش دهیم تا بتواند بهترین حرکت را برای گل‌زنی انجام دهد.
از آنجا که در صورتی که در زمان عدم تصاحب توپ بهترین حرکت به طور قطعی تعقیب توپ است، با کمک یادگیری تقویتی نحوه ضربه زدن به توپ برای فریب دروازه‌بان را بررسی می‌کنیم.
برای دروازه‌بان رقیب به طور پیش‌فرض از کد پایه‌ای که منشا اکثر تیم‌های امروزی است استفاده می‌کنیم؛ اما در نهایت اثر تغییر عامل دروازه‌بان را نیز بررسی می‌کنیم.
برای رسیدن به این هدف، قصد داریم محیط شبیه‌ساز دو بعدی فوتبال را به شکل یکی از محیط‌های معروف یادگیری تقویتی، یعنی محیط جیم\LTRfootnote{Gym}
که توسط شرکت اوپن‌ای‌آی\LTRfootnote{OpenAI} ارائه شده پیاده‌سازی کنیم.
\section{کار‌های مشابه}
بسیاری از تیم‌ها سالانه تلاش می‌کنند با کمک یادگیری ماشین\LTRfootnote{Machine Learning}
 و هوش مصنوعی، عامل‌های خود را بهبود بخشند.
در این بخش به برخی از این تلاش‌ها اشاره می‌کنیم.

تیم برین‌استرمرز\LTRfootnote{Brain Stormers}
که در سال ۱۹۹۸ تشکیل شد، یکی از قدیمی‌ترین تیم‌های لیگ شبیه‌ساز دو بعدی است که قصد داشته تمام بخش‌های کد خود را به کمک یادگیری تقویتی پیاده‌سازی کند.
این تیم در مراحل ابتدایی بخش‌های سطح پایین تیم خود را با کمک یادگیری تقویتی بهبود دادند، و به مرور زمان مسائل سطح بالا و تاکتیکی خود را نیز به سمت یادگیری تقویتی گسترش دادند\cite{riedmiller2001karlsruhe,riedmiller2005brainstormers}.

تیم اف‌آر‌ای-یونایتد\LTRfootnote{FRA-United}
که باقی‌مانده تیم برین‌استرمرز و وارث کد‌های آن است، سالیانه در گزارش فنی خود یک عمل را برای بهبود به کمک یادگیری تقویتی انتخاب می‌کند\cite{gabel2019fra}.
این تیم یک زیرساخت خاص‌منظوره دارد که به کمک آن جمع‌آوری داده و تکرار آزمایش‌ها را به سادگی انجام می‌دهد.
متاسفانه این زیرساخت‌های خاص‌منظوره خصوصی اند و در دسترس عموم نیستند.

پروژه متن‌باز\LTRfootnote{Open Source}
تهاجم نیمه میدانی\LTRfootnote{Half Field Offense}
یک پروژه بر پایه محیط شبیه‌ساز دو بعدی است که با کمک آن می‌توان یک تیم مهاجم و یک تیم مدافع را اجرا کرد\cite{kalyanakrishnan2007half,hausknecht2016half}.
این پروژه به منظور توسعه الگوریتم‌های یادگیری تقویتی برای تیم‌های مهاجم و مدافع ایجاد شده‌است، و امکان نوشتن عامل به کمک زبان پایتون\LTRfootnote{Python}
 و سی‌پلاس‌پلاس\LTRfootnote{C++}
را فراهم می‌کند.
برای انجام این پروژه این می‌شد تیم‌ها را با یک مهاجم و مدافع اجرا کرد، اما در فصل‌های آینده با معایب و مشکل‌های این پروژه بیشتر آشنا می‌شویم.
\section{ابزار‌ها و نرم‌افزار‌های مورد استفاده}
در این بخش به معرفی ابزار‌ها و نرم‌افزار‌هایی که در این پروژه استفاده شده‌اند، و نحوه استفاده از این ابزار‌ها می‌پردازیم.
\begin{itemize}
    \item \textbf{سرور مسابقات:}
    سرور مسابقات یک نرم‌افزار است که برای اجرای مسابقات ربوکاپ استفاده می‌شود.
    این نرم افزار وظیفه شبیه‌سازی عمل‌های بازیکنان، تشکیل حالت بازی و شبیه‌سازی فیزیک بازی را بر عهده دارد.
    در کنار این وظایف، این نرم‌افزار ارتباط با عامل‌ها را برقرار می‌کند و اطلاعات مربوط به حالت بازی را به عامل‌ها ارسال می‌کند.
    \item \textbf{نمایش‌گر مسابقات:}
    این نرم‌افزار وظیفه نمایش بازی‌های مسابقات را بر عهده دارد.
    این نرم‌افزار از اطلاعاتی که از سرور مسابقات دریافت می‌کند برای نمایش حالت بازی استفاده می‌کند.
    \item \textbf{کد پایه ایجنت دو بعدی\LTRfootnote{Agent2D}:}
    این کد پایه یک کد پایه برای عامل‌های مسابقات دو بعدی است که در سال ۲۰۱۲ توسط هیدهیسا آکیاما از تیم هلیوس ارائه شد\cite{akiyama2014helios}.
    این کد پایه شامل توابعی برای اتصال به سرور، تصمیم‌گیری برای حرکت، ضربه زدن به توپ، تکل زدن، و ... است.
    این کد پایه به صورت متن‌باز در دسترس تیم‌های مختلف است و می‌تواند به عنوان یک نقطه شروع برای توسعه عامل‌های جدید استفاده شود.
    \item \textbf{سی‌پلاس‌پلاس:}
    از آنجا که کد پایه ایجنت به زبان سی‌پلاس‌پلاس نوشته شده‌است، برای توسعه و تغییرات در این کد پایه نیاز به استفاده از این زبان را داریم.
    \item \textbf{پایتون:} 
    با توجه به محبوبیت زبان برنامه نویسی پایتون و امکانات بالای آن برای پیاده‌سازی الگوریتم‌های یادگیری ماشین، از این زبان برنامه نویسی برای پیاده‌سازی محیط آموزش عامل مهاجم استفاده می‌کنیم.
    در این پروژه از پایتون نسخه \lr{۳.۱۱} استفاده شده‌است.
    \item \textbf{جی‌آر‌پی‌سی\LTRfootnote{gRPC}:}
    این ابزار یک روش اجرای کد از راه دور است. از این رو که جی‌آر‌پی‌سی مستقل از زبان است، می‌توان از آن برای ارسال پیام و فراخوانی توابع بین دو برنامه به زبان‌های مختلف استفاده کرد.
    در ادامه توضیح خواهیم داد که چگونه به این کمک این ابزار محیط پایتونی‌ای که برای آموزش عامل مهاجم استفاده می‌کنیم، را به محیط سی‌پلاس‌پلاسی که برای اجرای مسابقات استفاده می‌کنیم، متصل کرد.
    \item \textbf{کتاب‌خانه چند رشته‌ای\LTRfootnote{Threading}:}
    از این کتاب‌خانه برای اجرای چندین رشته همزمان در پایتون استفاده می‌کنیم.
    از این رو به این کتاب‌خانه نیاز داریم که حداقل یک رشته برای پاسخ درخواست‌های سرور و یک رشته برای اجرای الگوریتم‌های یادگیری تقویتی داشته باشیم.
    در فصل‌های بعدی به طور دقیق با زیرساخت پیاده‌سازی و اتصال این دو بخش به یکدیگر خواهیم پرداخت.
    \item \textbf{محیط جیم\LTRfootnote{Gym}:}
    این محیط یک رابط استاندارد برای آموزش عامل‌های یادگیری تقویتی است\cite{brockman2016openai}.
    از ابزار‌های مختلفی برای پیاده‌سازی محیط‌های آموزشی و ارتباط با عامل‌ها پشتیبانی می‌کند.
    به کمک این ابزار می‌توان محیط‌های آموزشی متنوعی را پیاده‌سازی کرد و از الگوریتم‌های یادگیری تقویتی مختلف برای آموزش عامل‌ها استفاده کرد، به گونه‌ای که محیط جیم تقریبا 
    در یادگیری تقویتی به صورت همگانی به عنوان یک محیط استاندارد شناخته شده است، و معروف‌ترین محیط‌های تحقیقاتی مانند
    ماشین تپه‌نورد\LTRfootnote{Mountain Car}\cite{Moore90efficientmemory-based}،
    بازی‌های آتاری\LTRfootnote{Atari Games}،
    و آونگ معکوس شده\LTRfootnote{Inverted Pendulum}،
    در این محیط پیاده‌سازی شده‌اند.
    از این محیط برای پیاده‌سازی محیط آموزش عامل مهاجم استفاده می‌کنیم.
    \item \textbf{کتاب‌خانه پای‌تورچ\LTRfootnote{PyTorch}:}
    این کتاب‌خانه یکی از کتاب‌خانه‌های معروف برای پیاده‌سازی شبکه‌های عصبی و الگوریتم‌های یادگیری عمیق است.
    از این کتاب‌خانه برای پیاده‌سازی مدل‌های یادگیری تقویتی و شبکه‌های عصبی در این پروژه استفاده می‌کنیم.
    پیاده‌سازی از پیش آماده شده گرادیان و بهینه‌ساز‌های مختلف، و امکان استفاده از پردازنده‌های گرافیکی برای محاسبات سریع، از ویژگی‌های این کتاب‌خانه است\cite{paszke2019pytorch}.
    \item \textbf{کتاب‌خانه استیبل بیس‌لاینز\LTRfootnote{Stable Baselines}:}
    این کتاب‌خانه یک کتاب‌خانه معروف برای پیاده‌سازی الگوریتم‌های یادگیری تقویتی است\cite{raffin2021stable}.
    این کتاب‌خانه از الگوریتم‌های معروفی همچون \lr{PPO}، \lr{A2C}، \lr{DQN}، و ... پشتیبانی می‌کند.
    مدل‌ها به صورت مستقل پیاده‌سازی شده اند، اما برای نمایش قدرت پیاده‌سازی یک محیط استاندارد یادگیری تقویتی، از این کتاب‌خانه نیز استفاده می‌کنیم.
    \item \textbf{کتاب‌خانه تنسوربورد\LTRfootnote{TensorBoard}:}
    این کتاب‌خانه یک ابزار بسیار رایج برای ذخیره‌سازی اطلاعات و ویژگی‌های مدل‌های یادگیری عمیق در حین آموزش است.
    رایج است این کتاب‌خانه را با کتاب‌خانه تنسورفلو\LTRfootnote{textbf{TensorFlow}} استفاده کرد؛ اما در این پرو‌ژه از آن به صورت مستقل استفاده می‌کنیم.
    \item \textbf{کتاب‌خانه مت‌پلات‌لیب\LTRfootnote{Matplotlib}:}
    از این کتاب‌خانه برای رسم نمودار‌ها و نمایش داده‌ها استفاده می‌کنیم.
    
\end{itemize}