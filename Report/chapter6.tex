\chapter{جمع‌بندی، نتیجه‌گیری و پیشنهادات}
در این فصل مروری بر سیر مطالب عنوان شده خواهیم‌داشت.
سپس مروری بر نتایج‌ و دستاورد‌های این پروژه خواهیم داشت و در نهایت پیشنهاداتی برای ادامه‌ی کار در این موضوع ارائه خواهیم کرد.
\section{جمع‌بندی و نتیجه‌گیری}
در این پروژه قصد داشتیم تا با استفاده از روش‌های یادگیری ماشینی، یک محیط مجازی برای آموزش و تقویت بازیکنان فوتبال ایجاد کنیم.
به این منظور ابتدا با مفاهیم اساسی یادگیری تقویتی آشنا شدیم،
با مهم‌ترین روش‌های پیاده‌سازی کلاسیک آن آشنا شدیم و در نهایت با روش‌های پیشرفته‌تری همچون یادگیری تقویتی عمیق و برخی الگوریتم‌های آن آشنا شدیم.
در دو الگوریتم یادگیری کیو عمیق و یادگیری گرادیان سیاست قطعی عمیق به طور دقیق با فرآیند یادگیری و عوامل موثر در آن آشنا شدیم.

در ادامه محیط شبیه‌ساز دو بعدی فوتبال را معرفی کردیم و برخی از ویژگی‌های آن را توضیح دادیم.
رفتار‌های مجاز بازیکنان را بررسی کردیم، و کمی با تاریخچه ربوکاپ آشنا شدیم. سپس حالت پنالتی را به عنوان یک محیط مناسب برای یادگیری تقویتی معرفی کردیم.
در بخش بعدی تلاش‌های متفاوتی که برای پیاده‌سازی یادگیری تقویتی در شبیه‌سازی فوتبال انجام شده‌بود را بررسی کردیم.
در نهایت رابط استاندارد جیم را معرفی کردیم و یک پیاده‌سازی برای اتصال این رابط با کد پایه شبیه‌ساز به کمک جی‌آر‌پی‌سی انجام دادیم.

در نهایت با استفاده از الگوریتم‌های یادگیری تقویتی عمیق، یک مدل برای آموزش سمت مهاجم در تک به تک با دروازه‌بان ایجاد کردیم و الگوریتم‌های متفاوت را مقایسه کرده، و پیشنهاداتی برای بهبود عملکرد آن‌ها ارائه کردیم.
\section{کار‌های آتی}
در آینده می‌توان از زیرساخت ایجاد شده در سه راستا استفاده کرد: حل سایر مسائل در لیگ شبیه‌ساز دو بعدی، استفاده از محیط برای ارزیابی الگوریتم‌ها و روش‌های یادگیری،
و بررسی تاثیر ویژگی‌های مختلف بر عملکرد مدل‌های یادگیری تقویتی.
برخی از پیشنهاداتی که در این زمینه‌ها می‌توان ارائه کرد عبارتند از:
\begin{enumerate}
    \item   بررسی یادگیری تقویتی در حالت‌های چند عامله: یکی از بزرگترین چالش‌های یادگیری تقویتی، یادگیری در حالت‌های چند عامله‌ای است که نیاز به هماهنگی بین عامل‌ها دارد. این زیرساخت و محیط فوتبال می‌تواند محل مناسبی برای توسعه ایده‌های جدید در این راستا باشد.

    یک نمونه از این مسائل انتخاب حریف و یارگیری به صورت مجزا برای هر بازیکن است. از چالش‌های این مسئله می‌توان به مشکل همگام بودن عوامل در تصمیم‌گیری، بزرگی فضای حالت، و تفاوت میان اطلاعات عامل‌ها به دلیل دید ناقص اشاره کرد.
    \item بررسی تغییر روش نمایش ویژگی‌ها و تاثیر آن بر عملکرد مدل‌های یادگیری تقویتی: یکی از مهم‌ترین مسائل در یادگیری تقویتی، انتخاب نمایش مناسب برای ویژگی‌ها است. انتخاب نمایش مناسب می‌تواند تاثیر بزرگی بر عملکرد مدل‌ها داشته باشد.
    به طور مثال، موقعیت بازیکن را می‌توان به صورت دکارتی یا قطبی از مبدا مرکز زمین، مرکز دروازه، دروازه‌بان یا توپ نسبت به بازیکن دیگر نمایش داد. بررسی تاثیر و اهمیت نحوه نمایش ویژگی‌ها بر عملکرد مدل‌ها می‌تواند موضوعی جالب برای تحقیقات آینده باشد.

    همچنین افزودن برخی ویژگی‌ها مانند سرعت بازیکنان یا انرژی بازیکن می‌تواند تاثیر مهمی روی عملکرد تیم داشته‌باشد، اگرچه ممکن است به طلسم ابعاد\LTRfootnote{Curse of Dimensionality} بر بخوریم.
    \item استفاده از روش‌هایی همچون آموزش انتقالی\LTRfootnote{Transfer Learning} برای افزایش سرعت آموزش:
    همانطور که در نمودار‌های یادگیری مانند شکل \ref{fig:discretization_change}
    می‌توان دید، عامل زمان زیادی را در ابتدای آموزش صرف جستجوی بخش‌های بیهوده فضای حالت می‌کند. استفاده از یادگیری انتقالی می‌تواند کمک کند از عامل‌های از پیش نوشته‌شده استفاده کنیم تا روند آموزش را سریع‌تر کنیم.
    \item استفاده از روش‌های یادگیری متقابل برای آموزش دروازه‌بان و مهاجم به صورت همزمان:
    می‌توان از روش‌های یادگیری متقابل برای آموزش دو بازیکن به صورت همزمان استفاده کرد. این روش‌ها می‌توانند بهبود عملکرد هر دو بازیکن را به صورت همزمان و بهینه کنند.
    به کمک این روش‌ها می‌توان همزمان دروازه‌بان و مهاجمی ساخت که هر دو از یادگیری دیگری بهره‌مند شوند و از بهترین تیم‌ها هم قوی‌ترند.
\end{enumerate}
%%%%%%%%%%%%%%%%%%%%%%%%%%%%%%%%%%%%%%%%%%%
% در بخش پایانی گزارش، جمعبندي و مروري بر سیر مطالب عنوان شده در گزارش خواهیم داشت. همچنین نتایج حاصل را بیان
% کرده و پیشنهادهایی براي ادامه کار در این موضوع ارائه میدهیم.
% \section{نتیجه‌گیری}
% این گزارش ۶ الگوریتم‌ برای پیش‌بینی مصرف انرژی ساختمان‌ها را مورد بررسی قرار داد که برای این الگوریتم‌ها داده‌های مصرف آینده‌ی ساختمان‌ها و مصرف گذشته‌شان موجود بود. 
% همچنین در این مقاله اهمیت برخی متغیرها مانند وضعیت دمای محیط و تعداد ساکنین ساختمان در هر زمان نیز مورد توجه قرار گرفتند و اهمیت آن‌ها در تصمیم گیری مشخص گردید.
% چندین مدل یادگیری ماشین موفق با استفاده از داده‌های انرژی ثبت‌شده گذشته برای پیش‌بینی کوتاه‌مدت، میان‌مدت و بلندمدت توسعه یافته‌اند. مشاهده می شود 
% که هر یک از تکنیک های توصیف شده دارای مجموعه ای از مزایا و معایب است. اینها با توجه به تجزیه و تحلیل داده های انرژی ساختمان به تفصیل تجزیه و تحلیل و ارائه شده اند.
%  تأکید ویژه بر مدل ترکیبی داده شده است، که ترکیبی از دو یا چند تکنیک یادگیری ماشینی است به نحوی که هر مدل قدرت دیگری 
%  را تحسین می کند. به عنوان مثال، یک مدل ترکیبی که میانگین متحرک خودهمبسته یکپارچه و الگوریتم‌های تکاملی را در نظر می‌گیرد، می‌تواند
%   از مدل میانگین متحرک خودهمبسته یکپارچه برای تعیین تناوب و خطی بودن استفاده کند، در حالی که الگوریتم‌ تکاملی می‌تواند به طور موثر باقیمانده‌ها را تعیین کند. ترکیبات مختلفی از مدل ترکیبی و تازگی آنها در
%    ادبیات شناسایی شده و به طور سیستماتیک در این مقاله ارائه شده است. مشاهده می‌شود
%    که ترکیب تکنیک‌های پیش‌بینی سری‌های زمانی مانند شبکه ی عصبی مصنوعی، میانگین متحرک خودهمبسته یکپارچه به خوبی با 
%    تکنیک‌های بهینه‌سازی ترکیب می‌شوند. چنین ترکیب‌هایی به طور گسترده در تحقیقات اختصاص یافته به بهینه‌سازی ساختمان مورد بررسی قرار گرفته‌اند.
%     انتظار می رود روند رو به رشد
%     در تحقیقات در بهره وری انرژی ساختمان در پرتو انگیزه پایداری جهانی ادامه یابد. این امر نظارت و پیش بینی داده های انرژی در 
%     زمان واقعی را در این زمینه مرتبط و حیاتی می کند. این مقاله خلاصه‌ای جامع از تکنیک‌های پیش‌بینی موجود همراه با ترکیبی
%      از مدل ترکیبی ارائه می‌کند و راه را برای تحقیقات آینده در زمینه مصرف انرژی ساختمان هموار می‌کند.
%      \\
%      حوزه بهینه سازی ساختمان بر اساس یک شبکه گسترده جمع آوری داده، نظارت، پیش بینی، بهینه سازی و کنترل است. تمام این زیرساخت‌ها
%       همواره به کل هزینه عملیاتی ساختمان می‌افزایند. چالش در اینجا بررسی مزایای اضافه شده از نظر هزینه سرمایه گذاری و هزینه به دست آمده
%       به دلیل صرفه جویی در انرژی ناشی از بهینه سازی ساختمان است. مطالعات کمی وجود دارد که بخش مالی را برای کنترل عملکرد و بهینه سازی ساختمان 
%      برجسته می کند. این محدودیت در به اشتراک گذاری داده های هزینه یا به دلیل 
%      نیروهای بازار درگیر است یا به دلیل محرمانه بودن ماهیت داده های درگیر. لبی الدان\LTRfootnote{Labeodan} 
%      و همکاران. کاربردهای شبکه حسگرها و محرک‌های بی‌سیم کم‌هزینه را برای مدل‌سازی اشغال و کنترل روشنایی در یک ساختمان
%       اداری مورد بحث قرار داد \cite{labeodan2016application}. هزینه کل سیستم تقریبا 2575 یورو برای 12 ایستگاه کاری بود که
%       شامل حسگرهای حرکتی بی سیم و سنسورهای صندلی بود. نتایج نشان می دهد که به طور متوسط 24\% کاهش در مصرف انرژی روشنایی برای یک دوره
%       دو هفته ای با هزینه اجرا شده در حدود 215 یورو برای هر ایستگاه کاری است. نویسندگان خاطرنشان کردند که هزینه اولیه بالاتر و عدم آگاهی از عوامل مؤثر
%       در کاهش سرعت استقرار حسگرها هستند. با این حال، صرفه جویی در انرژی به دست آمده، سهولت استقرار و بهبود سنجش محیطی این
%       را به عنوان یک راه حل مناسب برای دستیابی به عملکرد بهبود 
%       یافته ساختمان نشان می دهد. کومار و همکاران همچنین متوجه شد که هزینه سنسورهای نظارت کنترل کیفیت هوای داخلی الزامات استقرار در مقیاس بزرگ برای کنترل و اتوماسیون را
%        برآورده نمی کند \cite{kumar2016real}. لیلیس و همکاران اشاره کنید
%        که علاقه به راه‌حل‌های 
%       مبتنی بر اینترنت اشیا\LTRfootnote{Internet Of Things (IOT)} مدرن برای بهینه‌سازی ساختمان به دلیل عدم برآورد منافع هزینه، مهار شده است \cite{lilis2017towards}. چن و همکاران اشاره کرد
%        که هزینه مربوط به مصرف انرژی مجموعه تصادفی ساختمان‌ها در چین با سیستم های اتوماسیون ساختمان تقریبا دو برابر ساختمان های
%        بدون سیستم اتوماسیون ساختمان است \cite{chen2016cost}. این به دلیل نقص سنسور و نقص استراتژی کنترل است که منجر به افزایش قابل توجهی در مصرف انرژی نهایی می شود.
%         \\
%         چند مطالعه مروری شبکه حسگر بی سیم را پوشش داده است که سیستم مدیریت انرژی ساختمان را برای کاربردهای خانه/ساختمان 
%         هوشمند فعال کرده است \cite{kazmi2014review,kuzlu2015review}. از آنجایی که فناوری کنترل پیش‌بینی مدل هنوز در مرحله توسعه است و 
%         نیاز به بهینه‌سازی سنگین بر اساس نوع و عملکرد ساختمان دارد،
%          پیاده‌سازی در حال حاضر بیشتر بر روی بستر آزمایشی و اعتبارسنجی تمرکز دارد. در عین حال، این فناوری هنوز
%          با هزینه لازم برای استقرار در مقیاس های بزرگ در دسترس نیست. 
%          این چالش ها منجر به نفوذ آهسته اتوماسیون ساختمان و بهینه سازی در کل می شود. دامنه این مقاله مروری، با این حال
%         ، محدود به مطالعه تکنیک‌های پیش‌بینی سری‌های زمانی برای مصرف انرژی ساختمان است که بخشی جدایی‌ناپذیر
%          از فرآیند بهینه‌سازی و کنترل ساختمان است.
% \section{پیشنهادها}
%     به طور کلی الگوریتم‌های هوش‌ مصنوعی که در زمینه‌ی پیش‌بینی سری داده‌های زمانی کار میکنند و بخش عمده‌ی آن‌ها که الگوریتم‌های یادگیری ماشین می‌شوند مشکلات و سختی‌های مخصوصی دارند
%     از جمله محدودیت‌های مربوط به یادگیری‌آن‌ها که نیازمند داده‌های بسیتر زیاد برای یادگیری و آموزش میباشد و هم‌چنین نیاز به توان پردازشی بالای آن‌ها که از جمله مشکلات روش‌های هوش مصنوعی 
%     به طور کلی میباشد. 
%     برای حل این مشکل پیشنهاد میشود که الگوریتم‌های نوینی با استفاده از روش‌های ترکیبی توسعه داده بشوند که نیاز به داده‌های زیاد برای یادگیری در آن‌ها کمتر باشد و بتوانند با شبیه‌سازی 
%     آموزش ببینند و همچنین برای حل مشکل پردازش‌های سنگین با تحقیقات جدید به سمت برخط کردن یادگیری الگوریتم‌های هوش مصنوعی حرکت بکنیم به این صورت که تمام پردازش‌های مورد نیاز 
%     الگوریتممان برروی سرورهای قدرتمندی در سطح منطقه انجام بشوند و ساختمان‌ها مشکل پردازشی‌شان از این نظر مرتفع بشود و تنها برای انجام اندازه‌ی مشخصی از پردازش برروی شبکه ی قدرتمندمان 
%     هزینه پرداخت کنند تا هزینه‌هایشان کاهش پیدا بکند.