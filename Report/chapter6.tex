\chapter{جمع‌بندی، نتیجه‌گیری و پیشنهادات}
در این فصل مروری بر سیر مطالب عنوان شده خواهیم داشت.
سپس مروری بر نتایج‌ و دستاورد‌های این پروژه خواهیم داشت و در نهایت پیشنهاداتی برای ادامه‌ی کار در این موضوع ارائه خواهیم کرد.
\section{جمع‌بندی و نتیجه‌گیری}
در این پروژه قصد داشتیم تا با استفاده از روش‌های یادگیری تقویتی یک محیط مجازی برای آموزش و تقویت بازیکنان فوتبال ایجاد کنیم.
به این منظور ابتدا با مفاهیم اساسی یادگیری تقویتی آشنا شدیم،
با مهم‌ترین روش‌های پیاده‌سازی کلاسیک آن آشنا شدیم و در نهایت با روش‌های پیشرفته‌تری همچون یادگیری تقویتی عمیق و برخی الگوریتم‌های آن آشنا شدیم.
در دو الگوریتم یادگیری کیو عمیق و یادگیری گرادیان سیاست قطعی عمیق به طور دقیق با فرآیند یادگیری و عوامل موثر در آن آشنا شدیم.

در ادامه محیط شبیه‌ساز دو بعدی فوتبال را معرفی کردیم و برخی از ویژگی‌های آن را توضیح دادیم.
رفتار‌های مجاز بازیکنان را بررسی کردیم، و کمی با تاریخچه ربوکاپ آشنا شدیم. سپس حالت پنالتی را به عنوان یک محیط مناسب برای یادگیری تقویتی معرفی کردیم.
در بخش بعدی تلاش‌های متفاوتی که برای پیاده‌سازی یادگیری تقویتی در شبیه‌سازی فوتبال انجام شده‌بود را بررسی کردیم.
در نهایت رابط استاندارد جیم را معرفی کردیم و یک پیاده‌سازی برای اتصال این رابط با کد پایه شبیه‌ساز به کمک جی‌آر‌پی‌سی انجام دادیم.

در نهایت با استفاده از الگوریتم‌های یادگیری تقویتی عمیق، یک مدل برای آموزش سمت مهاجم در موقعیت‌های تک به تک با دروازه‌بان ایجاد کردیم و الگوریتم‌های متفاوت را مقایسه کرده، و پیشنهاداتی برای بهبود عملکرد آن‌ها ارائه کردیم.
\section{کار‌های آتی}
در آینده می‌توان از زیرساخت ایجاد شده در سه راستا استفاده کرد: حل سایر مسائل در لیگ شبیه‌ساز دو بعدی، استفاده از محیط برای ارزیابی الگوریتم‌ها و روش‌های یادگیری،
و بررسی تاثیر ویژگی‌های مختلف بر عملکرد مدل‌های یادگیری تقویتی.
برخی از پیشنهاداتی که در این زمینه‌ها می‌توان ارائه کرد عبارتند از:
\begin{enumerate}
    \item   بررسی یادگیری تقویتی در حالت‌های چند عامله: یکی از بزرگترین چالش‌های یادگیری تقویتی، یادگیری در حالت‌های چند عامله‌ای است که نیاز به هماهنگی بین عامل‌ها دارد. این زیرساخت و محیط فوتبال می‌تواند محل مناسبی برای توسعه ایده‌های جدید در این راستا باشد.

    یک نمونه از این مسائل انتخاب حریف و یارگیری به صورت مجزا برای هر بازیکن است. از چالش‌های این مسئله می‌توان به مشکل همگام بودن عوامل در تصمیم‌گیری، بزرگی فضای حالت، و تفاوت میان اطلاعات عامل‌ها به دلیل دید ناقص اشاره کرد.
    \item بررسی تغییر روش نمایش ویژگی‌ها و تاثیر آن بر عملکرد مدل‌های یادگیری تقویتی: یکی از مهم‌ترین مسائل در یادگیری تقویتی، انتخاب نمایش مناسب برای ویژگی‌ها است. انتخاب نمایش مناسب می‌تواند تاثیر بزرگی بر عملکرد مدل‌ها داشته باشد.
    به طور مثال، موقعیت بازیکن را می‌توان به صورت دکارتی یا قطبی از مبدا مرکز زمین، مرکز دروازه، دروازه‌بان یا توپ نسبت به بازیکن دیگر نمایش داد. بررسی تاثیر و اهمیت نحوه نمایش ویژگی‌ها بر عملکرد مدل‌ها می‌تواند موضوعی جالب برای تحقیقات آینده باشد.

    همچنین افزودن برخی ویژگی‌ها مانند سرعت بازیکنان یا انرژی بازیکن می‌تواند تاثیر مهمی روی عملکرد تیم داشته‌باشد، اگرچه ممکن است به نفرین ابعاد\LTRfootnote{Curse of Dimensionality} بر بخوریم.
    \item استفاده از روش‌هایی همچون یادگیری انتقالی\LTRfootnote{Transfer Learning} برای افزایش سرعت آموزش:
    همانطور که در نمودار‌های یادگیری مانند شکل \ref{fig:discretization_change}
    می‌توان دید، عامل زمان زیادی را در ابتدای آموزش صرف جستجوی بخش‌های بیهوده فضای حالت می‌کند. استفاده از یادگیری انتقالی می‌تواند کمک کند از عامل‌های از پیش نوشته‌شده استفاده کنیم تا روند آموزش را سریع‌تر کنیم.
    \item استفاده از روش‌های یادگیری متقابل برای آموزش دروازه‌بان و مهاجم به صورت همزمان:
    می‌توان از روش‌های یادگیری متقابل برای آموزش دو بازیکن به صورت همزمان استفاده کرد. این روش‌ها می‌توانند بهبود عملکرد هر دو بازیکن را به صورت همزمان بهینه کنند.
    به کمک این روش‌ها می‌توان همزمان دروازه‌بان و مهاجمی ساخت که هر دو از یادگیری دیگری بهره‌مند شوند و از بهترین تیم‌ها هم قوی‌ترند.
\end{enumerate}