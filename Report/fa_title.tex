%% -!TEX root = AUTthesis.tex
% در این فایل، عنوان پایان‌نامه، مشخصات خود، متن تقدیمی‌، ستایش، سپاس‌گزاری و چکیده پایان‌نامه را به فارسی، وارد کنید.
% توجه داشته باشید که جدول حاوی مشخصات پروژه/پایان‌نامه/رساله و همچنین، مشخصات داخل آن، به طور خودکار، درج می‌شود.
%%%%%%%%%%%%%%%%%%%%%%%%%%%%%%%%%%%%
% دانشکده، آموزشکده و یا پژوهشکده  خود را وارد کنید
\faculty{دانشکده مهندسی کامپیوتر}
% گرایش و گروه آموزشی خود را وارد کنید
\department{}
% عنوان پایان‌نامه را وارد کنید
\fatitle{استفاده از يادگيري تقويتي براي گل زدن در موقعيت تک به تک در فوتبال }
% نام استاد(ان) راهنما را وارد کنید
\firstsupervisor{دکتر احسان ناظرفرد}
% \secondsupervisor{}
% نام استاد(دان) مشاور را وارد کنید. چنانچه استاد مشاور ندارید، دستور پایین را غیرفعال کنید.
%\firstadvisor{نام کامل استاد مشاور}
%\secondadvisor{استاد مشاور دوم}
% نام نویسنده را وارد کنید
\name{آراد }
% نام خانوادگی نویسنده را وارد کنید
\surname{فیروزکوهی}
%%%%%%%%%%%%%%%%%%%%%%%%%%%%%%%%%%
\thesisdate{فروردین ۱۴۰۳}

% چکیده پایان‌نامه را وارد کنید
\fa-abstract{
    این پایان‌نامه به اکشتاف عمیق یادگیری تقویتی و نحوه عملکرد آن در گل زدن در تک به تک فوتبال می‌پردازد.
     ابتدا، ما به بررسی مفاهیم یادگیری تقویتی و اصول اساسی آن می‌پردازیم تا درکی جامع از چگونگی تصمیم‌گیری و یادگیری ماشین در محیط‌های پویا ارائه شود.
     سپس پلتفرم شبیه‌ساز دو بعدی فوتبال ربوکاپ که از آن قرار است استفاده کنیم، معرفی می‌شود.
     در این پژوهش، الگوریتم‌های مختلف یادگیری تقویتی از جمله شبکه \lr{Q}
      عمیق، یادگیری تقویتی مبتنی بر سیاست، و دیگر رویکردهای پیشرفته بررسی شده‌اند تا تأثیر آن‌ها بر بهبود عملکرد بازیکنان در زدن گل‌ها ارزیابی گردد.
      بخش مهمی از این تحقیق به ایجاد یک روش استاندارد برای انجام یادگیری تقویتی در شبیه‌سازی فوتبال ربوکاپ با استفاده از چارچوب جیم اختصاص یافته است، که نوید بخش یکپارچه‌سازی و استانداردسازی روش‌های یادگیری تقویتی در این حوزه است.
        در نهایت خواهیم دید که چگونه استفاده از این الگوریتم‌ها می‌تواند به طور موثری به بازیکنان کمک کنند تا در موقعیت‌های تک به تک بهترین تصمیم‌ها را بگیرند و گل‌های بیشتری بزنند.
        علاوه بر این، پایان‌نامه به بررسی تأثیر پارامترها و تنظیمات مختلف الگوریتمی بر عملکرد یادگیری تقویتی می‌پردازد و راهکارهایی برای بهبود عملکرد ارائه می‌دهد.
}

% کلمات کلیدی پایان‌نامه را وارد کنید
\keywords{یادگیری تقویتی، یادگیری تقویتی عمیق، شبیه‌ساز دوبعدی فوتبال، جیم، تک به تک با دروازه‌بان}



\AUTtitle{a}
%%%%%%%%%%%%%%%%%%%%%%%%%%%%%%%%%%
\vspace*{7cm}
\thispagestyle{empty}
\begin{center}
\includegraphics[height=5cm,width=12cm]{besm}
\end{center}